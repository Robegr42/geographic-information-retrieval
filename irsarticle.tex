\documentclass{irsarticle}

\usepackage{url}
\usepackage[utf8]{inputenc}
\usepackage{graphicx}


\begin{document}
\title{\bf Recuperación de información geográfica}
\author{Jorge J. Morgado Vega\qquad Roberto García Rodríguez}
\date{}
\maketitle

\institution{Facultad de Matemática y Computación, Universidad de la Habana}

\begin{abstract}
	Uno de los sub campos de la Recuperación de Información (RI) es la
	Recueración de Información Geográfica (RIG). Los sistemas de RIG son de
	gran utilidad en diferentes esferas de la sociedad, a nivel gubernamental,
	empresarial, personal, etc. En este trabajo se realizará un análisis del
	estado del arte de la RIG. Se abordarán algunas de las arquitecturas
	implementadas, la evaluación de estos sistemas, sus aplicaciones, los
	beneficios y limitantes así como desafíos futuros en este campo
	investigativo.
\end{abstract}

\renewcommand\abstractname{Abstract}

\begin{abstract}
	One of the subfields of Information Retrieval (IR) is Geographic Information
	Retrieval (GIR). GIR systems are very useful in different spheres of society,
	government, business, personal, etc. In this work, an analysis of the state of
	the art of RIG will be carried out. Some of the implemented architectures, the
	evaluation of these systems, their applications, the benefits and limitations
	as well as future challenges in this research field will be addressed.
\end{abstract}

\keywords{recuperación de información, información geográfica}{information
retrieval, geographic information}

\newpage
\tableofcontents
\newpage

\section{Características del problema}\label{sec:prob-charact}

Los Sistemas de Recuperación de Información (SRI) tienen una amplia gama de
variantes, surgidos a partir de las distintas necesidades de la vida diaria y
problemas presentados en el desarrollo de la humanidad, cada vez más complejos
y con la necesidad de herramientas cada vez más fuertes que ayuden en la
solución de los mismos. 

La Recuperación de Información Geográfica (RIG) es una de estas variantes. RGI
surge de la unión de de los SRI y los Sistemas de Información Geográficos(SIG),
los cuales se enfovan en el desarrollo y uso de teorías, métodos, tecnología y
datos para comprender los procesos geográficos, relaciones y patrones
\cite{chang2016}.


La RGI son herramientas de búsqueda para buscar en la web, documentos
empresariales y búsqueda local móvil que combinan consultas tradicionales
basadas en texto con consultas de ubicación, como mapas o nombres de lugares.
Son técnicas para construir un sistema de aplicación que pueda indexar,
consultar, recuperar y navegar por la información georreferenciada. Se supone
que RGI puede comprender mejor el conocimiento geográfico contenido en la web
documentos y consultas de los usuarios, y proporcionar una respuesta más
satisfactoria a las necesidades de los usuarios.

\newpage

\section{Arquitecturas propuestas}\label{sec:arch}

\newpage


\section{Evaluación del sistema}\label{sec:eval}

\newpage


\section{Aplicaciones}\label{sec:appl}

La aplicación de estos sistemas cada vez son más frecuentes, en ámbitos tanto
cotideanos como científicos. Hoy en día, muchos de los problemas resueltos
están orientados espacialmente, lo que significa que cada vez más personas
necesitan utilizar información espacial, rápido acceso a estos medios,
softwares y aplicaiones fáciles de usar (\emph{user friendly} en inglés), que
requieran de conociemientos específicos ni formación especial.

A continuación algunas de estas aplicaciones.

\subsection{Búsqueda local móvil}\label{sec:mobile}

Quizás una de las aplicaciones más importantes que se le puede atribuir a la
RIG. En la actualidad, el uso de dispositivos móviles casi se ha vuelto una
necesidad en la vida de todo ser humano, y las búsquedas en INTERNET son las
acciones más frecuentes en la web. Preguntas como: \emph{?` Cúal es el
restaurante más cercano?, ?` Estoy cerca del teatro X ?} o quizás consultas
como: \emph{Gasolineras cerca, Barberías a menos de 1Km}. Todas estás
cuestiones deben ser resueltas usando RIG, brindando una respuesta acertada y
ahorrando tiempo al usuario \cite{teevan2011, lymberopoulos2011}. 

\subsection{Neogeografía}\label{sec:neogeo}

Aunque el uso de la Neogeografía no suelen se formales ni analíticos, esta
basada en el uso de técnicas y herramientas geográficas para actividades
personales o de interes especifico de un grupo de personas no expertas en el
tema.\cite{turner2006}

El monitoreo del medioambiente con el uso de información geográfica es una de
las ramas de trabajo que actualmente se encuentra en investigación
\cite{connors2012}. 

Se trata principalmente del análisis de datos brindados de forma voluntaria
usada por personas para crear sus mapas, los cuales responden a necesidades
específicas, ya sean medioambientales, económicas, politicas o de interes
social. \cite{harris2012}

\subsection{Geoweb}\label{sec:geoweb}

Geoweb o Web geográfica(\emph{Geographic Web} según su nombre en inglés)
describe como la información a la cual se accede en línea es cada vez más
relevante para el lugar donde uno se encuentra. Se puede encontrar usos
variados en este tema, desde la Economía Política con evidencias en las
práticas estatales\cite{leszczynski2012}; hasta sitios para el análisis
de relaciones cívicas ciudadanas\cite{johnson2015}.

Geoweb es la fusión de información geográfica (basada en la ubicación) con la
información general que está disponible en Internet. Los Geonavegadores son
una expresión de este concepto, siendo \emph{Google Maps} uno de los productos
estrella en la actualidad donde mediante el procesamiento de datos y la
indexación con contenido geográfico es uno de los servicios más preciados y
usados dentro de la Geoweb

El interés en una Geoweb ha avanzado por las nuevas tecnologías,
conceptos y productos, específicamente la popularización del posicionamiento
con GPS, siendo este uno de los aportes más preciados. GPS, otra maravilla de
la ciencia, se posiciona como un producto ampliamente usado, dejando claro la
importancia de los servicios geoweb.

\newpage

\section{Beneficios y limitantes}\label{sec:pros-and-cons}

La extracción de datos geográficos de manera general puede ser de gran utilidad
en diferentes campos de investigación e incluso en la vida cotidiana
\cite{artz2009}. Muchos son los beneficios y limitantes que existen al procesar
información espacial. En las siguientes secciones se realizará una descripción
de los puntos fuertes y débiles más relevantes en los sistemas de RIG.

\subsection{Análisis de patrones geográficos}\label{sec:patr}

El análisis de patrones geográficos puede ser fundamental para la administración
de una empresa, localidad, país, etc. Se pueden analizar todo tipo de patrones
como: taza de crimen por zonas, tráfico de vehículos por zonas, los diferentes
lenguajes que se hablen en una zona, censos, etc. De esta manera se puede, por ejemplo,
distribuir de una manera mejor diferentes recursos como: el nivel de seguridad
policial en algún barrio, dónde es mejor situar una tienda o carteles promocionales,
qué zonas priorizar en la protección de desamparados, etc.

Con la extracción de estos datos se puede conocer mejor no solo la situación
actual de una zona en cuanto a un determinado tema, sino también el cambio que
ha experimentado en el tiempo.

\subsection{Salud ambiental}\label{sec:pros}

Uno de los principales beneficios de la extracción correcta de datos
geográficos es para analizar los cambios ambientales \cite{scholten1991} en el
planeta, o algún area en específico (muy relacionado con el tema tratado en la
sección \ref{sec:patr}). Con estos datos se puede determinar el impacto de la
actividad humana en el ambiente, como por ejemplo la contaminación de la
atmósfera, la contaminación de los recursos hídricos, la contaminación de los
suelos, etc. Una recuperación efectiva de datos relacionados con la salud
ambiental es crucial para el estudio de los cambios ambientales pasados y
futuros.

\subsection{Toma de decisiones}\label{sec:deci}

Otro de los principales beneficios de la extracción de datos geográficos está
presente generalmente al tomar deciciones. Los ejemplos son vastos, pueden ir
desde niveles empresariales (ej. empresas que quiere ampliar su negocio a otras
zonas; empresas distribuidoras de paquetes que quiere seleccionar las mejores
rutas; empresas que se dediquen a la extracción de recuros naturales) así como
a niveles más personales (ej. buscar restaurantes cercanos, planificar viajes, etc).

Estas desiciones, sin importar su naturaleza, son muchas veces de gran impacto
para la persona o institución que las toma. En la mayoría de los casos,
las mismas pueden ahorrar tiempo y dinero.

\subsection{Principales limitantes}\label{sec:limit}

Muchas son las limitantes de los sitemas de recuperación de información
geográfica hoy en día \cite{purves2011,purves2004}. Entre los muchos
inconvenientes en este campo se pueden mencionar: ambiguedades toponómicas,
falta de estandarización para describir los datos, falta de una métrica de
clasificación efectiva para organizar los resultados recuperados de una
consulta, etc. En las próximas secciones se describirián estas limitantes de
manera detallada.

\subsubsection{Ambiguedades toponómicas}\label{sec:ambig}

Uno de los principales problemas en la recuperación de información, y en
específico a la información geográfica, es la amiguedad (en el caso abordado en
este artículo: la toponimia \cite{buscaldi2009}). Es muy común, a nivel global,
que algunos lugares no tengan un nombre propio, sino que sean nombrados por
alguna circunstancia, objeto o incluso el nombre de una persona. Entre los
diferentes tipos de ambiguedades se encuentran: lugares nombres de objetos (ej.
Granada, La Palma, Palo Seco, La Paz, Las Tunas, Perico); nombres de personas
(ej. San Martín, Santiago, St. Louis, Santa Marta, Artemisa), lugares nombrados
por alguna circunstancia (ej. Matanzas, Limonar, Nevada, La Mancha\footnote{La
teoría más extendida estipula que es un nombre heredado de una palabra árabe la
cual significaba ``lugar seco'' \cite{laMancha}.}); lugares con el mismo nombre
(ej. Versalles en Méjico y Francia, Santiago en Cuba y Chile\footnote{En este
caso la desambiguación está dada por el nombre oficial de ambas ciudades
(Santiago de Cuba y Santiago de Chile)}, Washington (el estado) y Washington (la
ciudad) en Estados Unidos).

\subsubsection{Falta de estandarización}\label{sec:estand}

Otro de los problemas claves en la recuperación de información geográfica es la
falta de estandarización de los datos \cite{deAndrade2014}. Como es sabido
existen diversas formas de guardar información sobre un lugar. Una forma simple
y efectiva es guardar las coordenadas geográficas (latitud y longitud), sin
embargo, no siempre es efectivo este método. Las coordenadas solo denotan un
lugar exacto, por tanto grandes áreas no pueden ser representadas con una
coordenada. Otra forma puede ser guardar el nombre del lugar, no obstante,
además de las ambiguedades que esto genera, surgen otros problemas. Muchos
lugares no son siempre son referidos exactamente por su nombre, por ejemplo: la
Gran Manzana (Manhattan), la Ciudad del Pecado (Las Vegas), la Ciudad del Sol
(Miami), la Llave del Golfo (Cuba), etc. Incluso, existen lugares que, por las
diversidad de lenguajes que existen en la zona, tienen dos nombre diferentes.

No obstante a esta limitación, se han hecho algunos esfuerzos por mejorar la
representación de datos geográficos. Entre ellos se encuentran: GML un
\emph{markup language} dedicado a la representación de dominios geográficos y
OWL un lenguaje de representación de ontología genérico. Sin embargo, estos
lenguajes aún no pueden representar de forma completa los datos geográficos
\cite{abdelmoty2005}.

\subsubsection{Falta de una métrica de clasificación efectiva}\label{sec:metric}

Una de las partes más importantes de la recuperación de información es la
clasificación efectiva de los datos, y con ellos, datos geográficos
\cite{purves2004,mandl2008,cai2011}. El análisis de similitudes entre los datos
es uno de los factores claves en la extracción correcta de esta información
\cite{janowicz2011}. Sin embargo, el problema de clasificar información
geográfica es mucho más dinámico que solo dar un resultado basado en
similitudes textuales o espaciales entre datos de una fuente (ej. servicios
móviles basados en la ubicación del usuario) \cite{kumar2011}.

Al clasificar datos geográficos, muchos son los factores que dificultan la
clasificación como: el lugar donde se realiza la búsqueda (ej. cuando se quiere
encontrar alguna cafetería cercana), la fuente de donde se extraen los datos,
los factores vistos en las secciones \ref{sec:ambig} y \ref{sec:estand}, etc.

A pesar de estas limitantes varios estudios se han realizado sobre la evaluación
de la efectividad de los diversos algoritmos de clasificación de datos geográficos
 \cite{larson2004jul, larson2004sep}.

\newpage

\section{Desafíos futuros}\label{sec:chall}

~




\begin{thebibliography}{}

\bibitem{chang2016}
Chang, K. T. (2016). Geographic information system. International Encyclopedia
of Geography: People, the Earth, Environment and Technology: People, the Earth,
Environment and Technology, 1-9.

\bibitem{teevan2011}
Teevan, J., Karlson, A., Amini, S., Brush, A. B., \& Krumm, J. (2011, August).
Understanding the importance of location, time, and people in mobile local
search behavior. In Proceedings of the 13th international conference on human
computer interaction with mobile devices and services (pp. 77-80).

\bibitem{leszczynski2012}
Leszczynski, A. (2012). Situating the geoweb in political economy.
Progress in human geography, 36(1), 72-89.

\bibitem{johnson2015}
Johnson, P. A., Corbett, J. M., Gore, C., Robinson, P., Allen, P., \& Sieber,
R. (2015). A web of expectations: Evolving relationships in community
participatory Geoweb projects. ACME: An International Journal for Critical
Geographies, 14(3), 827-848.

\bibitem{lymberopoulos2011}
Lymberopoulos, D., Zhao, P., Konig, C., Berberich, K., \& Liu, J.
(2011, October). Location-aware click prediction in mobile local search. In
Proceedings of the 20th ACM international conference on Information and
knowledge management (pp. 413-422).

\bibitem{turner2006}
Turner, A. (2006). Introduction to neogeography. " O'Reilly Media, Inc.".

\bibitem{connors2012}
Connors, J. P., Lei, S., \& Kelly, M. (2012). Citizen science in the age of
neogeography: Utilizing volunteered geographic information for environmental
monitoring. Annals of the Association of American Geographers, 102(6),
1267-1289.

\bibitem{harris2012}
Harris, T. M., \& Lafone, H. F. (2012). Toward an informal spatial data
infrastructure: voluntary geographic information, neogeography, and the role
of citizen sensors. SDI, Communities and Social Media, 8, 8-12.

\bibitem{Kornai2005}
Kornai, A. (2005, September). Evaluating geographic information retrieval. In
Workshop of the Cross-Language Evaluation Forum for European Languages (pp.
928-938). Springer, Berlin, Heidelberg.

\bibitem{bors2000}
Bors, N. ‘Information retrieval, experimental models and
statistical analysis’. Journal of Documentation, vol 56, nº 1 January 2000. p.
71-90

\bibitem{artz2009}
ARTZ, M. (2009, September 14). Top Five Benefits of GIS. GIS AND SCIENCE.
\url{https://gisandscience.com/2009/09/14/top-five-benefits-of-gis/}

\bibitem{scholten1991}
Scholten, H. J., \& De Lepper, M. J. (1991). The benefits of the application of
geographical information systems in public and environmental health. World
health statistics quarterly. Rapport trimestriel de statistiques sanitaires
mondiales, 44(3), 160-170.

\bibitem{gey2007}
Gey, F., Larson, R., Sanderson, M., Bischoff, K., Mandl, T., Womser-Hacker, C.,
... \& Ferro, N. (2007). Challenges to evaluation of multilingual geographic
information retrieval in GeoCLEF. In quot; In Workshop on Evaluation of
Information Access (EVIA) May 15 (Tokyo Japan May 15 2007).

\bibitem{jones2008}
Jones, C. B., \& Purves, R. S. (2008). Geographical information retrieval.
International Journal of Geographical Information Science, 22(3), 219-228.

\bibitem{martins2005}
Martins, B., Silva, M. J., \& Chaves, M. S. (2005, November). Challenges and
resources for evaluating geographical IR. In Proceedings of the 2005 workshop
on Geographic Information Retrieval (pp. 65-69).

\bibitem{purves2011}
Purves, R., \& Jones, C. (2011). Geographic information retrieval. SIGSPATIAL
Special, 3(2), 2-4.

\bibitem{purves2018}
Purves, R. S., Clough, P., Jones, C. B., Hall, M. H., \& Murdock, V. (2018).
Geographic information retrieval: Progress and challenges in spatial search of
text. Foundations and Trends in Information Retrieval, 12(2-3), 164-318.

\bibitem{purves2014}
Purves, R. (2014). Geographic information retrieval: are we making progress. In
NCGIA specialist meeting on spatial search (pp. 1-6).

\bibitem{mandl2008}
Mandl, T., Gey, F., Nunzio, G. D., Ferro, N., Sanderson, M., Santos, D., \&
Womser-Hacker, C. (2008). An evaluation resource for Geographical Information
Retrieval. In quot; In Proceedings of the 6 th International Conference on
Language Resources and Evaluation (LREC 2008)(Marrakech 28-30 May 2008)
European Language Resources Association (ELRA). European Language Resources
Association (ELRA).

\bibitem{buscaldi2009}
Buscaldi, D. (2009, July). Toponym ambiguity in geographical information
retrieval. In Proceedings of the 32nd international acm sigir conference on
research and development in information retrieval (pp. 847-847).

\bibitem{cai2011}
Cai, G. (2011). Relevance ranking in geographical information retrieval.
SIGSPATIAL Special, 3(2), 33-36.

\bibitem{purves2004}
Purves, R., \& Jones, C. (2004, December). Workshop on geographic information
retrieval, SIGIR 2004. In ACM SIGIR Forum (Vol. 38, No. 2, pp. 53-56). New
York, NY, USA: ACM.

\bibitem{kumar2011}
Kumar, C. (2011, August). Relevance and ranking in geographic information
retrieval. In Fourth BCS-IRSG Symposium on Future Directions in Information
Access (FDIA 2011) 4 (pp. 2-7).

\bibitem{janowicz2011}
Janowicz, K., Raubal, M., \& Kuhn, W. (2011). The semantics of similarity in
geographic information retrieval. Journal of Spatial Information Science, (2),
29-57.

\bibitem{larson2004sep}
Larson, R. R., \& Frontiera, P. (2004, September). Spatial ranking methods for
geographic information retrieval (GIR) in digital libraries. In International
Conference on Theory and Practice of Digital Libraries (pp. 45-56). Springer,
Berlin, Heidelberg.

\bibitem{larson2004jul}
Larson, R. R., \& Frontiera, P. (2004, July). Ranking and representation for
geographic information retrieval. In Extended abstract in SIGIR 2004 Workshop
on Geographic Information Retrieval.

\bibitem{li1994}
Li, R. (1994). Data structures and application issues in 3-D geographic
information systems. Geomatica, 48(3), 209-224.

\bibitem{deAndrade2014}
de Andrade, F. G., de Souza Baptista, C., \& Davis, C. A. (2014). Improving
geographic information retrieval in spatial data infrastructures.
GeoInformatica, 18(4), 793-818.

\bibitem{abdelmoty2005}
Abdelmoty, A. I., Smart, P. D., Jones, C. B., Fu, G., \& Finch, D. (2005). A
critical evaluation of ontology languages for geographic information retrieval
on the Internet. Journal of Visual Languages \& Computing, 16(4), 331-358.

\bibitem{brown1999}
Brown, I. (1999). Developing a virtual reality user interface (VRUI) for
geographic information retrieval on the Internet. Transactions in GIS, 3(3),
207-220.

\bibitem{martins2007}
Martins, B., Borbinha, J., Pedrosa, G., Gil, J., \& Freire, N. (2007, November).
Geographically-aware information retrieval for collections of digitized
historical maps. In Proceedings of the 4th ACM Workshop on Geographical
information Retrieval (pp. 39-42).

\bibitem{laMancha}
Castilla-La Mancha. (2021, October 26). Wikipedia. Retrieved October 30, 2021,
from \url{https://es.wikipedia.org/wiki/Castilla-La_Mancha#Toponimia}.

\bibitem{purves2007} Purves, R. S., Clough, P., Jones, C. B., Arampatzis, A., Bucher, B.,
Finch, D., ... \& Yang, B. (2007). The design and implementation of SPIRIT: a
spatially aware search engine for information retrieval on the Internet.
International journal of geographical information science, 21(7), 717-745.

\bibitem{perea2007} Perea-Ortega, J. M., Cumbreras, M. Á. G., Vega, M. G., \&
Montejo-Ráez, A. (2007). GEOUJA System. University of Jaén at GeoCLEF 2007. In
CLEF (Working Notes).

\bibitem{bordogna2012} Bordogna, G., Ghisalberti, G., \& Psaila, G. (2012). Geographic
information retrieval: Modeling uncertainty of user's context. Fuzzy Sets and
Systems, 196, 105-124.

\end{thebibliography}

\end{document}
