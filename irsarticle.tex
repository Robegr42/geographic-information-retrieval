\documentclass{irsarticle}

\usepackage{url}
\usepackage[utf8]{inputenc}


\begin{document}
\title{\bf Recuperación de información geográfica}
\author{Jorge J. Morgado Vega\qquad Roberto García Rodríguez}
\date{}
\maketitle

\institution{Facultad de Matemática y computación, Universidad de la Habana}

\begin{abstract}
	Resumen del artículo
\end{abstract}

\keywords{recuperación de información, información geográfica}

\newpage
\tableofcontents
\newpage

\section{Características del problema}\label{sec:prob-charact}

Los Sistemas de Recuperación de Información (SRI) tienen una amplia gama de
variantes, surgidos a partir de las distintas necesidades de la vida diaria y
problemas presentados en el desarrollo de la humanidad, cada vez más complejos
y con la necesidad de herramientas cada vez más fuertes que ayuden en la
solución de los mismos. 

La Recuperación de Información Geográfica (RIG) es una de estas variantes. RGI
surge de la unión de de los SRI y los Sistemas de Información Geográficos(SIG),
los cuales se enfovan en el desarrollo y uso de teorías, métodos, tecnología y
datos para comprender los procesos geográficos, relaciones y patrones
\cite{chang2016}.


La RGI son herramientas de búsqueda para buscar en la web, documentos
empresariales y búsqueda local móvil que combinan consultas tradicionales
basadas en texto con consultas de ubicación, como mapas o nombres de lugares.
Son técnicas para construir un sistema de aplicación que pueda indexar,
consultar, recuperar y navegar por la información georreferenciada. Se supone
que RGI puede comprender mejor el conocimiento geográfico contenido en la web
documentos y consultas de los usuarios, y proporcionar una respuesta más
satisfactoria a las necesidades de los usuarios.

\newpage

\section{Arquitecturas propuestas}\label{sec:arch}

~


\section{Evaluación del sistema}\label{sec:eval}

~


\section{Aplicaciones}\label{sec:appl}

Las aplicaciones de estos sistemas cada vez son más frecuentes, en ámbitos
tanto cotideanos como científicos. Hoy en día, muchos de los problemas
resueltos están orientados espacialmente, lo que significa que cada vez más
personas necesitan utilizar información espacial, rápido acceso a estos medios,
softwares y aplicaiones fáciles de usar (\emph{user friendly} en inglés), que no
requieran de conociemientos específicos ni formación especial.

A continuación algunas de estas aplicaciones.

\subsection{Búsqueda local móvil}\label{sec:mobile}

Quizás una de las aplicaciones más importantes que se le puede atribuir a la
RIG. En la actualidad, el uso de dispositivos móviles casi se ha vuelto una
necesidad en la vida de todo ser humano, y las búsquedas en INTERNET son las
acciones más frecuentes en la web. Preguntas como: \emph{?` Cúal es el
restaurante más cercano?, ?` Estoy cerca del teatro X ?} o quizás consultas
como: \emph{Gasolineras cerca, Barberías a menos de 1Km}. Todas estás
cuestiones deben ser resueltas usando RIG, brindando una respuesta acertada y
ahorrando tiempo al usuario \cite{teevan2011, lymberopoulos2011}. 

\subsection{Neogeografía}\label{sec:neogeo}

Aunque los usos de la Neogeografía no suelen se formales ni analíticos, está
basada en el uso de técnicas y herramientas geográficas para actividades
personales o de interes especifico de un grupo de personas no expertas en el
tema.\cite{turner2006}

El monitoreo del medioambiente con el uso de información geográfica es una de
las ramas de trabajo que actualmente se encuentra en investigación
\cite{connors2012}. 

Se trata principalmente del análisis de datos brindados de forma voluntaria
usada por personas para crear sus mapas, los cuales responden a necesidades
específicas, ya sean medioambientales, económicas, políticas o de interés
social. \cite{harris2012}

\subsection{Geoweb}\label{sec:geoweb}

Geoweb o Web geográfica(\emph{Geographic Web} según su nombre en inglés)
describe como la información a la cual se accede en línea es cada vez más
relevante para el lugar donde uno se encuentra. Se puede encontrar usos
variados en este tema, desde la Economía Política con evidencias en las
práticas estatales\cite{leszczynski2012}; hasta sitios para el análisis
de relaciones cívicas ciudadanas\cite{johnson2015}.

Geoweb es la fusión de información geográfica (basada en la ubicación) con la
información general que está disponible en Internet. Los Geonavegadores son
una expresión de este concepto, siendo \emph{Google Maps} uno de los productos
estrella en la actualidad donde mediante el procesamiento de datos y la
indexación con contenido geográfico es uno de los servicios más preciados y
usados dentro de la Geoweb

El interés en una Geoweb ha avanzado por las nuevas tecnologías,
conceptos y productos, específicamente la popularización del posicionamiento
con GPS, siendo este uno de los aportes más preciados. GPS, otra maravilla de
la ciencia, se posiciona como un producto ampliamente usado, dejando claro la
importancia de los servicios geoweb.

\newpage

\section{Ventajas y desventajas}\label{sec:pros-and-cons}

~


\section{Desafíos futuros}\label{sec:chall}

~




\begin{thebibliography}{}

\bibitem{chang2016}
Chang, K. T. (2016). Geographic information system. International Encyclopedia of Geography: People, the Earth, Environment and Technology: People, the Earth, Environment and Technology, 1-9.

\bibitem{gey2007}
Gey, F., Larson, R., Sanderson, M., Bischoff, K., Mandl, T., Womser-Hacker, C.,
... \& Ferro, N. (2007). Challenges to evaluation of multilingual geographic
information retrieval in GeoCLEF. In quot; In Workshop on Evaluation of
Information Access (EVIA) May 15 (Tokyo Japan May 15 2007).

\bibitem{jones2008}
Jones, C. B., \& Purves, R. S. (2008). Geographical information retrieval.
International Journal of Geographical Information Science, 22(3), 219-228.

\bibitem{martins2005}
Martins, B., Silva, M. J., \& Chaves, M. S. (2005, November). Challenges and
resources for evaluating geographical IR. In Proceedings of the 2005 workshop
on Geographic Information Retrieval (pp. 65-69).

\bibitem{purves2011}
Purves, R., \& Jones, C. (2011). Geographic information retrieval. SIGSPATIAL
Special, 3(2), 2-4.

\bibitem{purves2018}
Purves, R. S., Clough, P., Jones, C. B., Hall, M. H., \& Murdock, V. (2018).
Geographic information retrieval: Progress and challenges in spatial search of
text. Foundations and Trends in Information Retrieval, 12(2-3), 164-318.

\bibitem{purves2014}
Purves, R. (2014). Geographic information retrieval: are we making progress. In
NCGIA specialist meeting on spatial search (pp. 1-6).

\bibitem{mandl2008}
Mandl, T., Gey, F., Nunzio, G. D., Ferro, N., Sanderson, M., Santos, D., \&
Womser-Hacker, C. (2008). An evaluation resource for Geographical Information
Retrieval. In quot; In Proceedings of the 6 th International Conference on
Language Resources and Evaluation (LREC 2008)(Marrakech 28-30 May 2008)
European Language Resources Association (ELRA). European Language Resources
Association (ELRA).

\bibitem{buscaldi2009}
Buscaldi, D. (2009, July). Toponym ambiguity in geographical information
retrieval. In Proceedings of the 32nd international acm sigir conference on
research and development in information retrieval (pp. 847-847).

\bibitem{cai2011}
Cai, G. (2011). Relevance ranking in geographical information retrieval.
SIGSPATIAL Special, 3(2), 33-36.

\bibitem{purves2004}
Purves, R., \& Jones, C. (2004, December). Workshop on geographic information
retrieval, SIGIR 2004. In ACM SIGIR Forum (Vol. 38, No. 2, pp. 53-56). New
York, NY, USA: ACM.

\bibitem{kumar2011}
Kumar, C. (2011, August). Relevance and ranking in geographic information
retrieval. In Fourth BCS-IRSG Symposium on Future Directions in Information
Access (FDIA 2011) 4 (pp. 2-7).

\bibitem{janowicz2011}
Janowicz, K., Raubal, M., \& Kuhn, W. (2011). The semantics of similarity in
geographic information retrieval. Journal of Spatial Information Science, (2),
29-57.

\bibitem{larson2004sep}
Larson, R. R., \& Frontiera, P. (2004, September). Spatial ranking methods for
geographic information retrieval (GIR) in digital libraries. In International
Conference on Theory and Practice of Digital Libraries (pp. 45-56). Springer,
Berlin, Heidelberg.

\bibitem{larson2004jul}
Larson, R. R., \& Frontiera, P. (2004, July). Ranking and representation for
geographic information retrieval. In Extended abstract in SIGIR 2004 Workshop
on Geographic Information Retrieval.

\bibitem{li1994}
Li, R. (1994). Data structures and application issues in 3-D geographic
information systems. Geomatica, 48(3), 209-224.

\bibitem{deAndrade2014}
de Andrade, F. G., de Souza Baptista, C., \& Davis, C. A. (2014). Improving
geographic information retrieval in spatial data infrastructures.
GeoInformatica, 18(4), 793-818.

\bibitem{abdelmoty2005}
Abdelmoty, A. I., Smart, P. D., Jones, C. B., Fu, G., \& Finch, D. (2005). A
critical evaluation of ontology languages for geographic information retrieval
on the Internet. Journal of Visual Languages \& Computing, 16(4), 331-358.

\bibitem{brown1999}
Brown, I. (1999). Developing a virtual reality user interface (VRUI) for
geographic information retrieval on the Internet. Transactions in GIS, 3(3),
207-220.

\bibitem{martins2007}
Martins, B., Borbinha, J., Pedrosa, G., Gil, J., \& Freire, N. (2007, November).
Geographically-aware information retrieval for collections of digitized
historical maps. In Proceedings of the 4th ACM Workshop on Geographical
information Retrieval (pp. 39-42).

\bibitem{laMancha}
Castilla-La Mancha. (2021, October 26). Wikipedia. Retrieved October 30, 2021,
from \url{https://es.wikipedia.org/wiki/Castilla-La_Mancha#Toponimia}.

\end{thebibliography}

\end{document}
