\section{Beneficios y limitantes}\label{sec:pros-and-cons}

La extracción de datos geográficos de manera general puede ser de gran utilidad
en diferentes campos de investigación e incluso en la vida cotidiana
\cite{artz2009}. Muchos son los beneficios y limitantes que existen al procesar
información espacial. En las siguientes secciones se realizará una descripción
de los puntos fuertes y débiles más relevantes en los sistemas de RIG.

\subsection{Análisis de patrones geográficos}\label{sec:patr}

El análisis de patrones geográficos puede ser fundamental para la administración
de una empresa, localidad, país, etc. Se pueden analizar todo tipo de patrones
como: taza de crimen por zonas, tráfico de vehículos por zonas, los diferentes
lenguajes que se hablen en una zona, censos, etc. De esta manera se puede, por ejemplo,
distribuir de una manera mejor diferentes recursos como: el nivel de seguridad
policial en algún barrio, dónde es mejor situar una tienda o carteles promocionales,
qué zonas priorizar en la protección de desamparados, etc.

Con la extracción de estos datos se puede conocer mejor no solo la situación
actual de una zona en cuanto a un determinado tema, sino también el cambio que
ha experimentado en el tiempo.

\subsection{Salud ambiental}\label{sec:pros}

Uno de los principales beneficios de la extracción correcta de datos
geográficos es para analizar los cambios ambientales \cite{scholten1991} en el
planeta, o algún area en específico (muy relacionado con el tema tratado en la
sección \ref{sec:patr}). Con estos datos se puede determinar el impacto de la
actividad humana en el ambiente, como por ejemplo la contaminación de la
atmósfera, la contaminación de los recursos hídricos, la contaminación de los
suelos, etc. Una recuperación efectiva de datos relacionados con la salud
ambiental es crucial para el estudio de los cambios ambientales pasados y
futuros.

\subsection{Toma de decisiones}\label{sec:deci}

Otro de los principales beneficios de la extracción de datos geográficos está
presente generalmente al tomar deciciones. Los ejemplos son vastos, pueden ir
desde niveles empresariales (ej. empresas que quiere ampliar su negocio a otras
zonas; empresas distribuidoras de paquetes que quiere seleccionar las mejores
rutas; empresas que se dediquen a la extracción de recuros naturales) así como
a niveles más personales (ej. buscar restaurantes cercanos, planificar viajes, etc).

Estas desiciones, sin importar su naturaleza, son muchas veces de gran impacto
para la persona o institución que las toma. En la mayoría de los casos,
las mismas pueden ahorrar tiempo y dinero.

\subsection{Principales limitantes}\label{sec:limit}

Muchas son las limitantes de los sitemas de recuperación de información
geográfica hoy en día \cite{purves2011,purves2004}. Entre los muchos
inconvenientes en este campo se pueden mencionar: ambiguedades toponómicas,
falta de estandarización para describir los datos, falta de una métrica de
clasificación efectiva para organizar los resultados recuperados de una
consulta, etc. En las próximas secciones se describirián estas limitantes de
manera detallada.

\subsubsection{Ambiguedades toponómicas}\label{sec:ambig}

Uno de los principales problemas en la recuperación de información, y en
específico a la información geográfica, es la amiguedad (en el caso abordado en
este artículo: la toponimia \cite{buscaldi2009}). Es muy común, a nivel global,
que algunos lugares no tengan un nombre propio, sino que sean nombrados por
alguna circunstancia, objeto o incluso el nombre de una persona. Entre los
diferentes tipos de ambiguedades se encuentran: lugares nombres de objetos (ej.
Granada, La Palma, Palo Seco, La Paz, Las Tunas, Perico); nombres de personas
(ej. San Martín, Santiago, St. Louis, Santa Marta, Artemisa), lugares nombrados
por alguna circunstancia (ej. Matanzas, Limonar, Nevada, La Mancha\footnote{La
teoría más extendida estipula que es un nombre heredado de una palabra árabe la
cual significaba ``lugar seco'' \cite{laMancha}.}); lugares con el mismo nombre
(ej. Versalles en Méjico y Francia, Santiago en Cuba y Chile\footnote{En este
caso la desambiguación está dada por el nombre oficial de ambas ciudades
(Santiago de Cuba y Santiago de Chile)}, Washington (el estado) y Washington (la
ciudad) en Estados Unidos).

\subsubsection{Falta de estandarización}\label{sec:estand}

Otro de los problemas claves en la recuperación de información geográfica es la
falta de estandarización de los datos \cite{deAndrade2014}. Como es sabido
existen diversas formas de guardar información sobre un lugar. Una forma simple
y efectiva es guardar las coordenadas geográficas (latitud y longitud), sin
embargo, no siempre es efectivo este método. Las coordenadas solo denotan un
lugar exacto, por tanto grandes áreas no pueden ser representadas con una
coordenada. Otra forma puede ser guardar el nombre del lugar, no obstante,
además de las ambiguedades que esto genera, surgen otros problemas. Muchos
lugares no son siempre son referidos exactamente por su nombre, por ejemplo: la
Gran Manzana (Manhattan), la Ciudad del Pecado (Las Vegas), la Ciudad del Sol
(Miami), la Llave del Golfo (Cuba), etc. Incluso, existen lugares que, por las
diversidad de lenguajes que existen en la zona, tienen dos nombre diferentes.

No obstante a esta limitación, se han hecho algunos esfuerzos por mejorar la
representación de datos geográficos. Entre ellos se encuentran: GML un
\emph{markup language} dedicado a la representación de dominios geográficos y
OWL un lenguaje de representación de ontología genérico. Sin embargo, estos
lenguajes aún no pueden representar de forma completa los datos geográficos
\cite{abdelmoty2005}.

\subsubsection{Falta de una métrica de clasificación efectiva}\label{sec:metric}

Una de las partes más importantes de la recuperación de información es la
clasificación efectiva de los datos, y con ellos, datos geográficos
\cite{purves2004,mandl2008,cai2011}. El análisis de similitudes entre los datos
es uno de los factores claves en la extracción correcta de esta información
\cite{janowicz2011}. Sin embargo, el problema de clasificar información
geográfica es mucho más dinámico que solo dar un resultado basado en
similitudes textuales o espaciales entre datos de una fuente (ej. servicios
móviles basados en la ubicación del usuario) \cite{kumar2011}.

Al clasificar datos geográficos, muchos son los factores que dificultan la
clasificación como: el lugar donde se realiza la búsqueda (ej. cuando se quiere
encontrar alguna cafetería cercana), la fuente de donde se extraen los datos,
los factores vistos en las secciones \ref{sec:ambig} y \ref{sec:estand}, etc.

A pesar de estas limitantes varios estudios se han realizado sobre la evaluación
de la efectividad de los diversos algoritmos de clasificación de datos geográficos
 \cite{larson2004jul, larson2004sep}.

\newpage
