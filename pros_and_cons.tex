\section{Beneficios y limitantes}\label{sec:pros-and-cons}

\subsection{Limitantes}\label{sec:limit}

Muchas son las limitantes de los sitemas de recuperación de información
geográfica hoy en día \cite{purves2011,purves2004}. Entre los muchos
inconvenientes en este campo se pueden mencionar: ambiguedades toponómicas,
falta de estandarización para describir los datos, falta de una métrica de
clasificación efectiva para organizar los resultados recuperados de una
consulta, etc. En las próximas secciones se describirián estas limitantes de
manera detallada.

\subsubsection{Ambiguedades toponómicas}\label{sec:ambig}

Uno de los principales problemas en la recuperación de información, y en
específico a la información geográfica, es la amiguedad (en el caso abordado en
este artículo: la toponimia \cite{buscaldi2009}). Es muy común, a nivel global,
que algunos lugares no tengas un nombre propio, sino que sean nombrados por
alguna circunstancia, objeto o inclusoo el nombre de una persona. Entre los
diferentes tipos de ambiguedades se encuentran: lugares nombres de objetos (ej.
Granada, La Palma, Palo Seco, La Paz, Las Tunas, Perico); nombres de personas
(ej. San Martín, Santiago, St. Louis, Santa Marta, Artemisa), lugares nombrados
por alguna circunstancia (ej. Matanzas, Limonar, Nevada, La Mancha\footnote{La
teoría más extendida estipula que es un nombre heredado de una palabra árabe la
cual significaba ``lugar seco'' \cite{laMancha}.}); lugares con el mismo nombre
(ej. Versalles en Méjico y Francia, Santiago en Cuba y Chile\footnote{En este
caso la desambiguación está dada por el nombre oficial de ambas ciudades
(Santiago de Cuba y Santiago de Chile)}, Washington (el estado) y Washington (la
ciudad) en Estados Unidos).

\subsubsection{Falta de estandarización}\label{sec:estand}

Otro de los problemas claves en la recuperación de información geográfica es la
falta de estandarización de los datos \cite{deAndrade2014}. Como es sabido
existen diversas formas de guardar información sobre un lugar. Una forma simple
y efectiva es guardar las coordenadas geográficas (latitud y longitud), sin
embargo, no siempre es efectivo este método. Las coordenadas solo denotan un
lugar exacto, por tanto grandes áreas no pueden ser representadas con una
coordenada. Otra forma puede ser guardar el nombre del lugar, no obstante,
además de las ambiguedades que esto genera, surgen otros problemas. Muchos
lugares no son siempre son referidos exactamente por su nombre, por ejemplo: la
Gran Manzana (Manhatan), la Ciudad del Pecado (Las vegas), la Ciudad del Sol
(Miami), la Llave del Golfo (Cuba), etc. Incluso, existen lugares que, por las
diversidad de lenguajes que existen en la zona, tienen dos nombre diferentes.

No obstante a esta limitación, se han hecho algunos esfuerzos por mejorar la
representación de datos geográficos. Entre ellos se encuentran: GML un
\emph{markup language} dedicado a la representación de dominios geográficos y
OWL un lenguaje de representación de ontología genérico. Sin embargo, estos
lenguajes aún no pueden representar de forma completa los datos geográficos
\cite{abdelmoty2005}.

\subsubsection{Falta de una métrica de clasificación efectiva}\label{sec:metric}

Una de las partes más importantes de la recuperación de información es la
clasificación efectiva de los datos, y con ellos, datos geográficos
\cite{purves2004,mandl2008,cai2011}. El análisis de similitudes entre los datos
es uno de los factores claves en la extracción correcta de esta información
\cite{janowicz2011}. Sin embargo, el problema de clasificar información
geográfica es mucho más dinámico que solo dar un resultado basado en
similitudes textuales o espaciales entre datos de una fuente (ej. servicios
móviles basados en la ubicación del usuario) \cite{kumar2011}.

Al clasificar datos geográficos, muchos son los factores que dificultan la
clasificación como: el lugar donde se realiza la búsqueda (ej. cuando se quiere
encontrar alguna cafetería cercana), la fuente de donde se extraen los datos,
los factores vistos en las secciones \ref{sec:ambig} y \ref{sec:estand}, etc.

A pesar de estas limitantes varios estudios se han realizado sobre la evaluación
de la efectividad de los diversos algoritmos de clasificación de datos geográficos
 \cite{larson2004jul, larson2004sep}.

