\section{Evaluación del sistema}\label{sec:eval}

La naturaleza determinista de los SRI propicia su necesidad intrínseca de
evaluación. Por ello y paralelamente al desarrollo de su tecnología, surge un
amplio campo de trabajo dedicado específicamente a la determinación de medidas
que permitan valorar su efectividad. Un repaso exhaustivo de la bibliografía
especializada permite identificar varios grupos de evaluaciones: las basadas
en la relevancia de los documentos, las basadas en los usuarios y un tercer
grupo de medidas alternativas a la realización de los juicios de relevancia,
que pretenden evitar afectarse de las dosis de subjetividad que estos juicios
poseen de forma inherente.

Los sistemas RGI, como cualquier otro sistema, son susceptibles de ser
sometidos a evaluación, con el fin de que sus usuarios se encuentren en
condiciones de valorar su efectividad y, de este modo, adquieran confianza
en los mismos.

El proceso de los sistemas RGI en principio son bastante parecidos a los 
SRI estándar, variando solo algunos pasos específicos que el propio
sistema impone dada las caracteristicas del mismo y del problema que pretende
resolver, como la detección de entidades geográficas. \cite{Kornai2005}

En este punto surge una nueva cuestión, si bien podría parecer trivial a
primera vista, puede marcar en gran medida el resultado de un proceso de
evaluación. Esta cuestión no es otra que responder con certeza a la pregunta
``?`cuándo la información es relevante?''.

El término relevancia significa ``calidad o condición de relevante,
importancia, significación'', y el término relevante lo define como
``importante o significativo''. Entendiéndose, por extensión de las definiciones
anteriores, que una información recuperada se puede considerar relevante
cuando el contenido de la misma posee alguna significación o importancia
con motivo de la pregunta realizada por el usuario, es decir, con su
necesidad de información.

Los primeros estudios de evaluación de los SRI datan casi del surgimiento de
los mismos, siendo estos los pioneros de la evaluación de los sistemas, los
cuales sirvieron de base para la creación de futuros métodos y análisis que
se fueron transformando a través del tiempo y adaptándose a los nuevos
sistemas que fueron apareciendo. Estos estudios son Cranfield,
MEDLARS o SMART. \cite{bors2000}


\newpage

