\section{Características del problema}\label{sec:prob-charact}

~
Los Sistemas de Recuperación de Información (SRI) tienen una amplia gama de
variantes, surgidos a partir de las distintas necesidades de la vida diaria y
problemas presentados en el desarrollo de la humanidad, cada vez más complejos
y con la necesidad de herramientas cada vez más fuertes que ayuden en la
solución de los mismos. 

La Recuperación de Información Geográfica (RIG) es una de estas variantes. RGI
surge de la unión de de los SRI y los Sistemas de Información Geográficos(SIG),
los cuales se enfovan en el desarrollo y uso de teorías, métodos, tecnología y
datos para comprender los procesos geográficos, relaciones y patrones
\cite{chang2016}.


La RGI son herramientas de búsqueda para buscar en la web, documentos
empresariales y búsqueda local móvil que combinan consultas tradicionales
basadas en texto con consultas de ubicación, como mapas o nombres de lugares.
Son técnicas para construir un sistema de aplicación que pueda indexar,
consultar, recuperar y navegar por la información georreferenciada. Se supone
que RGI puede comprender mejor el conocimiento geográfico contenido en la web
documentos y consultas de los usuarios, y proporcionar una respuesta más
satisfactoria a las necesidades de los usuarios.