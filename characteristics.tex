\section{Características del problema}\label{sec:prob-charact}

Los Sistemas de Recuperación de Información (SRI) tienen una amplia gama de
variantes, surgidos a partir de las distintas necesidades de la vida diaria y
problemas presentados en el desarrollo de la humanidad. A medida que las
necesidades de las personas van evolucionando así lo hacen los SRI. Es por ello
que cada vez sea más necesario que estos sistemas sean más rápidos y eficaces.

Uno de los subcampos de estos sistemas es la Recuperación de Información
Geográfica (RIG). La RGI surge de la unión de de los SRI y los Sistemas de
Información Geográficos (SIG), los cuales se enfocan en el desarrollo y uso de
teorías, métodos, tecnología y datos para comprender los procesos geográficos,
relaciones y patrones \cite{chang2016}.

Los sistemas de RGI son herramientas de búsqueda para buscar en la web,
documentos empresariales y búsqueda local móvil que combinan consultas
tradicionales basadas en texto con consultas de ubicación, como mapas o nombres
de lugares. Son técnicas para construir un sistema de aplicación que pueda
indexar, consultar, recuperar y navegar por la información georreferenciada. El
objetivo de los sistemas de RGI es comprender mejor el conocimiento geográfico
contenido en la web, documentos y consultas de los usuarios, y proporcionar una
respuesta más satisfactoria a las necesidades de los usuarios.

Una de las características propias de los sitemas de RGI es que en muchas
ocasiones la información obtenida puede ser visualizada de diferentes formas
(comúnmente en mapas). Esto permite a los usuarios tener una mejor
percepción de la información y una mejor comprensión de la misma.

Muchos son los otros campos científicos que están relacionados con la RIG:
el procesamiento de lenguaje natural para la comprención de textos no
estructurados y en las búsquedas que realizan los usuarios; la linguísitca en
el análisis de documentos, búsquedas o datos de forma general que se encuentran
en otros idiomas; entre muchos otros.

El presente trabajo está estructurado de la siguiente forma: en la sección
\ref{sec:arch} se mostrará la arquitectura de un sistema de RIG básico así
como alguna de las implementaciones que existen; en la sección \ref{sec:eval} se
analizará la evaluación de estos sistemas; en la sección \ref{sec:appl} se
describirán las diversas aplicaciones de los sistemas de RIG; en la sección
\ref{sec:pros-and-cons} se mostrarán los principales beneficios y limitantes
que existen al usar un sistema de RIG; y en la sección \ref{sec:chall} se
describirán los retos que se presentan en el desarrollo de los sistemas de RIG
en la actualidad.

\newpage
