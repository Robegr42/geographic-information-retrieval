\section{Aplicaciones}\label{sec:appl}

\newpage

Como se puede deducir las aplicaciones de este sistemas son tan variadas como
uno pueda imaginarse, resolviendo así miles de problemas tanto cotideanos como
como en el ámbito de la ciencia.

Hoy en día, muchos de los problemas resueltos están orientados espacialmente,
significa que cada vez más personas necesitan utilizar información espacial.
Deben poder utilizarlo rápidamente, sin ninguna herramienta de software
especial y sin ninguna formación especial. Por eso necesitan una solución
fácil de usar.

A continuación algunas de estas aplicaciones.

\subsection{Búsqueda local móvil}\label{sec:mobile}

Quizás una de las aplicaciones más importantes que se le puede atribuir a la
RIG. En la actualidad, el uso de dispositivos móviles casi se ha vuelto una
necesidad en la vida de todo ser humano, y las búsquedas en INTERNET son las
acciones más frecuentes en la web. Preguntas como: \emph{?` Cúal es el
restaurante más cercano?, ?` Estoy cerca del teatro X ?} o quizás consultas
como: \emph{Gasolineras cerca, Barberías a menos de 1Km}. Todas estás
cuestiones deben ser resueltas usando RIG, brindando una respuesta acertada y
ahorrando tiempo al usuario \cite{teevan2011, lymberopoulos2011}. 

\subsection{Neogeografía}\label{sec:neogeo}

\subsection{Geoweb}\label{sec:geoweb}

Geoweb o Web geográfica(\emph{Geographic Web} según su nombre en inglés)
describe como la información a la cual se accede en línea es cada vez más
relevante para el lugar donde uno se encuentra. Se puede encontrar usos
variados en este tema, desde la Economía Política con evidencias en las
práticas estatales\cite{leszczynski2012}; hasta sitios para el análisis
de relaciones cívicas ciudadanas\cite{johnson2015}.

Geoweb es la fusión de información geográfica (basada en la ubicación) con la
información general que está disponible en Internet. Los Geonavegadores son
una expresión de este concepto, siendo \emph{Google Maps} uno de los productos
estrella en la actualidad donde mediante el procesamiento de datos y la
indexación con contenido geográfico es uno de los servicios más preciados y
usados dentro de la Geoweb

El interés en una Geoweb ha avanzado por las nuevas tecnologías,
conceptos y productos, específicamente la popularización del posicionamiento
con GPS, siendo este uno de los aportes más preciados. GPS, otra maravilla de
la ciencia, se posiciona como un producto ampliamente usado, dejando claro la
importancia de los servicios geoweb.