\section{Desafíos futuros}\label{sec:chall}

Muchos son los desafíos que exiten en vías de mejorar la recuperación de datos
geográficos. Entre los más mencionados \cite{purves2014}, se enceuntran:
detectar referencias geográficas en forma de nombres y lugares, así como
referencias espaciales asociadas a calificadores de lenguaje natural en
documentos de textos o en búsquedas hechas por los usuarios; desambiguar nombre
de lugares; interpretar de forma geométrica nombres de lugares poco
descriptivos como ``Midlands'' y de referencias espaciales poco descriptivas
como ``cerca'', ``lejos'', etc.; indexar documentos respecto a el contenido
geográfico; desarrollar interfaces gráficas que faciliten el acceso a datos
geográficos; desarrollar métodos de evaluación para los diferentes sistemas de
GIR; entre otros.

En publicaciones más recientes \cite{purves2018}, se exponen como desafíos:
entender la importancia que tiene la información geográfica en diferents campos
inversigativos como las ciencias de la computación, geometría computacional,
linguística, etc.; la aplicación del \emph{machine learning} y \emph{deep
learning} en el procesamiento de lenguaje natural como vía para la
clasificación o recopilación de datos y referencias geográficas; la publicación
de métodos, algoritmos, conjunto de datos, y resultados reproducibles para así
poder realizar comparaciones en el estado del arte; en cuanto al indexado, no
solo realizarlo de una forma eficiente sino especificar correctamente qué se
está indexando; definir de una mejor forma la relevancia de datos extraídos;
entre otros.

En el mismo documento se hace referencia también a desafíos planteados con
anterioridad pero que todavía no han sido completamente desarrollados como: el
trabajo con datos vernáculos o históricos; vías más inteligentes para el
procesamiento del lenguaje espacial así como expresiones espaciales como:
``40km al norte de ...''; entre otros.

En las siguientes secciones se abordarán de forma más detallada algunos de los
desafíos enunciados posteriormente.

\subsection{Detección de referencias espaciales en textos}\label{sec:detect}

La extracción de locaciones geográficas en textos es uno de los desafíos más
importantes en la recuperación de datos geográficos. Especificamente, se plantea
no solo la extracción mediante la resolución de nombres directamente, sino
el reconocimiento de referencias espaciales. A modo de resumen: saber cuando una
información hace referencia a un lugar geográfico (aún cuando no se mencione
directamente). Esto es conocido como reconocmiento toponómico.

\subsection{Interpretación geométrica de nombres y referencias poco
descriptivas}\label{sec:geom}

En muchas ocaciones los nombres de lugares al ser poco conocidos, pueden no
encontrarse entre los nombres conocidos geograficamente hablando. Si a esto
se le suma el hecho de que estos nombres pueden ser poco descriptivos, entonces
la tarea de relacionarlos con una locación puede ser bien complicada. Además,
en muchos casos, los lugares son referenciados mediante otros lugares (ej. 
40km al norte de...) lo cual hace que sea aún más difícil la interpretación de
los mismos.

\subsection{Desarrollo de interfaces gráficas}\label{sec:gui}

Debido a la gran diveridad que existe para la representación de datos
geográficos el desarrollo de una interfáz gráfica cómoda, sencilla y eficiente
para el usuario es uno de los desafíos más importantes a tener en cuenta. De
nada sirve poder tener un buen modelo de GIR si no se puede entender de forma
correcta la búsqueda del usuaro, o si los resultados se muestran de manera
ineficiente. Según la bibliografía, mostrar los resultados de una búsqueda en
un sistema de GIR no ha avanzado mucho más que mostrar puntos en un mapa. Es
por ello que la geovisualización es un punto clave en los GIR.


\subsection{Importancia de la información geográfica en diferentes
campos}\label{sec:import}

Este punto puede no parecer un desafío, sin embargo, muchos no conocen o relacionan
la importancia que tiene el estudio de datos geograficos (así como su extracción)
en diferentes campos investigativos como: las ciencias de la computación, la 
geometría computacional, la linguística, siencias de la información, etc.

La recuperación de información geográfica va más allá del
reconocimiento de nombres de lugares, sino también del reconocimiento de referencias así
como el análisis e interpretación de las mismas en textos de una maneras más
sofisticadas. Es por ello que es considerado ineherentemente un trabajo
interdisciplinario.

\subsection{Aplicación del \emph{machine learning} y \emph{deep learning} en el
procesamiento de lenguaje natural}\label{sec:ml}

Diferentes sub-campos de la Inteligencia Artifical como el \emph{machine
learning} y \emph{deep learning} son unos de los más usados en la actualidad
para el procesamiendo de datos de forma general. Se plantea como desafío la
integración de estas herramientas en el procesamiento de lenguaje natural para
la recuperación de datos geográficos.

En muchos casos estas herramientas son entrenadas con un conjunto de datos que
contienen informacion adicional en forma de meta datos, lo cual no es un escenario
muy realista, donde normalmente la mayoria de la información está no estructurada.
Esto aumenta considerablemnte la complejidad del problema y lo hace un desafío más
en los sistemas de recuperación de información geográfica.

\subsection{Evaluación y publicación de métodos, algortimos, datos y
resultados reproducibles}\label{sec:publ}

A pesar de los avances realizados en este campo, uno de los desafíos más
importantes es la publicación de los mismos de manera reproducible, así como
los métodos, algoritmos y conjunto de datos usados. En muchos casos, los
resultados obtenidos no se encuentran bien documentados, o no hacen referencias
y comparaciones con otros método. Esto está ligado directamente con el hecho de
que la evaluación de los sistemas de recuperación de información geográfica todavía
es un punto débil en este campo.


\newpage

