\section{Conclusiones}\label{sec:conclusions}

Como se pudo apreciar en el trabajo presentado, la recuperación de información
geográfica es uno de los subcampos más complejos de la recuperación de datos
en general. Muchos han sido los avances realizados en esta área, pero también
hay muchos problemas y desafíos que quedan por resolver de manera más eficiente
y/o eficaz.

En el trabajo presentado de mostraron las características principales de los
sistemas de RIG. La arquitectura básica de estos sistemas así como algunas
variaciones implementadas fueron explicadas también. Se realizó un análisis de
los diferentes métodos de evaluación de los sistemas de RIG así como la
aplicación de los mismos en diferentes tipos de problemas. Se debatieron los
principales beneficios y limitantes que existen en el uso de estos sistemas.
Además, se mostró un resumen de los principales desafíos que están presentes
actualmente en el campo de la recuperación de información geográfica.


\section{Recomendaciones}\label{sec:recomend}

Se exhorta a los profesoras, estudiantes y otros investigadores a profundizar
en el tema de la recuperación de información geográfica. Es todavía un
campo que se está explorando y que necesita de la participación de todos
para su mejoramiento.

\newpage
